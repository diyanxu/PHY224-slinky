\documentclass{article}
\usepackage{graphicx}
\usepackage[utf8]{inputenc}
\usepackage{amsmath}

\title{Investigation into wave propagation through a slinky}
\author{Rickie Li, Di Yan Xu}
\date{\today}
\begin{document}

\maketitle

\newpage

\section{Introduction}
The purpose of this experiment was to understand wave propagation through the
medium of a slinky. This was done using a slinky suspended in the air by string
and a piston moving in sinusoidal motion. The slinky being suspended in the air
by strings creates a pendulum system in which the length of the string is $l$
and the weight of the pendulum is the mass of the slinky.\\
% I got no clue what to write here maybe after its all done
The system with this slinky is a dispersive system, that means the angular
frequency ($\omega$) and wave number ($k$) are releated in slightly more
complicated form than the following equations
\begin{align}
    \omega &= \sqrt{\omega_0^2 + c_0^2 k^2} \\
    k^2 &= \frac{\omega^2 - \omega_0^2}{c_0^2}
\end{align}
Where $\omega_0$ is the angular frequency for a single pendulum of the slinky
suspension string system and $c_0$ is, the constant wave speed of a
non dispersive system.\\
In this experiment we are observing different properties of wave propagation
through the slinky by running the motor at different values of $\omega$ relative
to the angular frequency of the slinky system.
% somthing about the piston moving at omega
% list some of the important -ish equations here

\section{Methods}
The following equipment was used for this experiment
\begin{itemize}
    \item[-] Slinky
    \item[-]Piston and Motor
    \item[-] String
    \item[-] Meter Stick
    \item[-] Filtering Combs 
    \item[-] Camera
\end{itemize}
Using setup provided we drive the motor at various different $\omega$ values
and using a camera we can find the $\omega$ by observing how long the motor
takes to make one rotation. We also use the meter stick to measure the various
features observed in the slinky as waves are propagated through.\\
For experiments 1 through 6, we were asked to determine and observe properties
of the slinky with no filter combs in the way. In experiments 7 through 11, We
lowered the filtering combs on certian sections of the slinky effectively
shortening the length of the string thus would increase the $\omega_0$ of the
shortened section.

\newpage
\subsection*{Experiments}

\subsubsection*{Experiment 1}
Experiment one consists of measuring the amount of time it takes for a wave to
travel 2 lengths of the slinky, this is achieved by recording the wave being
sent through the slinky and measuring the amount of time it takes to return.

\subsubsection*{Experiment 2}
Experiment two consists of determining the angular frequency of the slinky as a
pendulum, this is achieved by measuring the length of the string suspending the
slinky.

\subsubsection*{Experiment 3}
Experiment three consists of plotting $\omega^2$ with $k^2$, this is achieved by
running the piston at various speeds of $\omega$ then measuring the distance
between nodes. The experiment also asks to run a curve fit and determine
$\omega_0^2$ and $c_0^2$.

\subsubsection*{Experiment 4}
Experiment four consists of testing different values of resonant $\omega$ values
and determining if they are integer factors of some fundamental angular
frequency or if they are better modeled with the equation below.

\subsubsection*{Experiment 5}
Experiment five consists of driving the piston at $\omega << \omega_0$ and
taking rough approximations of the minimum and maximum positions of certain
coils along the slinky. The second part consists of diving the piston at
$\omega \rightarrow 0$ with the piston at a maximum and minimum and gathering
the same data.

\subsubsection*{Experiment 6}
Experiment six consists of driving the piston at $\omega = \omega_0$ and
repeating the data gathering in the first part of experiment 5.

\subsubsection*{Experiment 7}
Experiment seven consists of dropping the filtering combs onto the slinky and
computing the new $\omega_s$ for the shorter lengths of strings.

\subsubsection*{Experiment 8}
Experiment eight consists of lower filter comb numbers 2 and 3, then driving the
piston to try and determine the resonance belonging to just section 1.

\subsubsection*{Experiment 9}
Experiment nine consists of driving the piston at the same speed as in
experiment 8 and lowering filtering combs 1 and 2 to see if the wave can tunnel
into section 3 of the slinky. The experiment also asks to observe if section 3
exhibit larger amplitudes than sections 1 and 2.

\subsubsection*{Experiment 10}
Experiment ten consists of doing the same as experiment 8 then suddenly turning
off the motor and raising filtering comb 3 at the same time. The experiment then
wants to know if it is possible to determine the beat period of the tunneling.

\subsubsection*{Experiment 11}
Experiment eleven consists of driving the piston 10$\%$ faster and 10$\%$ slower
than one of the resonance speeds and observing the properties of the slinky.

\section{Analysis}
\subsubsection*{Find the $c_0$ of this system.}
We know that the $c_0$ of a system is the distance travelled over time. To
approximate the $c_0$ of this system, we used the camera to capture a video
and determine the time it takes to get from one end and back. To do this we
video editing software to go frame by frame to see when the wave started and
ended. After this we determined that the wave traveled the distance of the
slinky which is a distance of $3.78 \pm 0.01$ meters in $2.46 \pm 0.02$ seconds.
Since the software used only allows for seconds to end in $0.00,0.03, 0.06$ we
estimate error to be around $0.02s$. Using this data we can approximate $c_0$
to be
\begin{align*}
    c_0 &= \frac{3.78m}{2.46s}\\
    &= 1.536 \frac{m}{s}
\end{align*}
So we approximate $c_0 = 1.536 \pm 0.013 \frac{m}{s}$
\newpage

\subsubsection*{Deduce $\omega_0$ with the filtering combs out of the way.}
The $\omega_0$ of a pendulum system can be determined using the following
\begin{align*}
    \omega_0 &= \sqrt{\frac{g}{L}}
\end{align*}
We measured the length of the string to be $0.875 \pm 0.005$ so we can compute
$\omega_0$
\begin{align*}
    \omega_0 &= \sqrt{\frac{9.81\frac{m}{s^2}}{0.875}}\\
    &= 3.348 s^{-1}
\end{align*}
So we calculated $\omega_0 = 3.348 \pm 0.107 s^{-1}$ 

\subsubsection*{Plot $\omega^2$ against $k^2$ and deduce $\omega_0^2$ and $c^2_0$
from a linear fit}
After collecting 3 data points and determining the $k$ we were able to generate
the following plot (figure 1)\\
\begin{figure}[hp!]
    \centering
    \includegraphics[scale=0.6]{graphs/q3.png}
    \caption{}
\end{figure}
Running a curve fit on the equaiton $\omega^2, = \omega_0^2 + c_0^2k^2$, we get
a curve fit values of $\omega_0^2 = 4.27$ and $c_0^2 = -67.42$. These value
aren't close (one is even negative)
to the calculated values we got for $\omega_0^2$ and $c_0^2$, this may be due
to errors in data collection.
\newpage

\subsubsection*{Find the family of resonant $\omega$ values within range of
values provided by driving motor}
After testing different values provided by out driving motor, we determined that
the resonant values within the range of values in our driving motor are around
$4.05 \pm 0.02s^{-1}, 5.98 \pm 0.11 s^{-1}, 7.85 \pm 0.19s^{-1}$.Looking at
these values they are not integer values of some
fundanmental angular frequency, rather they are more generally releated to the
equation below for values of $n = 1,2,3$
\begin{align*}
    \omega &= \sqrt{\omega_0^2 + k_n^2c_0^2}\\
    k_n &= \frac{n\pi}{L}
\end{align*}
If we were to use our values for $c_0 = 1.536 \pm 0.013 \frac{m}{s}, w_0 =
3.348\pm0.019 s^{-1}$ and $L = 1.89\pm 0.005m$, we would get for $n = 1, 2, 3$,
$\omega = 4.21 \pm 0.02s^{-1}, 6.11 \pm 0.039s^{-1}, 8.35 \pm 0.062s^{-1}$.

\subsubsection*{Drive the slinky at an $\omega\ll\omega_{o}$.  
Does the amplitude fall off exponentially with distance
from the driven end?  Or are both exponentials in $y=y_{0}sin\omega 
t(e^{+kx}-e^{-kx})$ required to describe it?}
When the slinky is driven by the motor with its dial at 10\% which translates 
to $\omega$ of approximately $0.650\pm0.001s^{-1}$, we see in figure 2 
that the amplitude does seem to fall off exponentially with distance from
the driven end ($x=0$ is at the stationary end).  It seems that having both
exponentials is not much different from having the one exponential, where 
using both exponentials gives a $\chi_{red}^{2}$ value of approximately 
0.0440 while one exponential gives a $\chi_{red}^{2}$ value of approximately 
0.0427.

\subsubsection*{Try the extreme of $\omega\rightarrow0$ by merely displacing 
and holding one end stationary and noting the displacements $y$ of the coils 
down the apparatus.}
Using the data obtained by recording the positions of coils down the apparatus 
when the driven end is at its maximum and minimum position, we get the graph in
figure 3  Like in the previous part, it seems that both one and two 
exponentials can describe the amplitude and are not much different.  The 
$\chi_{red}^{2}$ corresponding to the one exponential model is 0.123 and the
$\chi_{red}^{2}$ for the two exponential model is 0.139.

\begin{figure}[hp!]
    \centering
    \includegraphics[scale=0.45]{graphs/driven_10.png}
    \caption{}
    \centering
    \includegraphics[scale=0.45]{graphs/driven_0.png}
    \caption{}
    \centering
    \includegraphics[scale=0.45]{graphs/driven_40.png}
    \caption{}
\end{figure}
\newpage
\subsubsection*{Drive the slinky at $\omega=\omega_{0}$.  Collect  some  data  and 
perform a linear fit to test how accurately the linear decay law, 
$y=y_{d}\frac{x}{L}\sin{\omega t}$, is obeyed.  Discrepancies will indicate 
imperfections of suspension, or non-uniformities in the slinky, or perhaps other 
dissipative effects.  Can you distinguish between them?}
After collecting the data and performing a linear fit to test the linear decay law, 
we see in figure 4 that the linear decay law appears to be obeyed.
Regarding discrepancies, we noticed that the slinky did not make a straight line
from the driven end to the stationary end and also did not line up with the 
driven end.  The slinky was also somewhat stretched out away from the driven end.

\subsubsection*{Let $\omega_{l}$ be the $\omega_{0}$ for the original long 
strings and $\omega_{s}$ be the $\omega_{0}$ for the short strings, with 
$\omega_{l}<\omega_{s}$.}
We measured that the length of the combed strings were approximately 
$0.210\pm0.005m$.  Using this value we find that, 
\begin{align*}
    \omega_{s}&=\sqrt{\frac{9.81\frac{m}{s^2}}{0.210m}}\\
    &=6.84s^{-1}
\end{align*}
Where $\omega_{s}$ is the angular frequency for a single pendulum of the slinky
system in the short section.  The error is,
\begin{align*}
    \sigma_{s}&=\omega_{s}\frac{\sigma_{l_{s}}}{2l_{s}}\\
    &=6.84s^{-1}\frac{0.005m}{0.420m}\\
    &=0.08s^{-1}
\end{align*}
Where $\sigma_{l_{s}}$ is the error in the length of the short strings, and 
$l_{s}$ is the length of the short strings.  So, we calculated 
$\omega_{s}=6.84\pm0.08s^{-1}$.

\subsubsection*{Try lowering combs number 2 and number 3, away from the driven 
end.  Now you can establish a resonance belonging essentially to section number 1, 
nearest the drive.}
Resonance in section number 1 seems to have been established at around 90\% on 
the motor's dial which translates to an $\omega$ of approximately 
$8.27\pm0.2s^{-1}$.  We found this by first using the equation, 
\[\omega=\sqrt{\omega_{0}^2+\frac{c_{0}^2\pi^2}{L_l^2}}\]
With our previously acquired value of $\omega_{0}$ and $c_{0}$ as well as the
length of the uncombed section, $L_l$, which was approximately $0.590\pm0.005m$
in order to get an estimate for around where we might get resonance in the 
section number 1.
\newpage
\subsubsection*{Perform experiment 9 and observe the results, does section 3
exhibit resonance, does section 3 have a larger amplitude than sections 1 and 2
despite being further from the driven end?}
Yes, section 3 does seem to exhibit resonance with amplitudes longer than
sections 1 and 2 despite being further away from the driven end. One possible
reason for this is that sections 1 and 2 require a higher $\omega$ to experience
resonance since the $\omega$ of the driven piston is lower than the $\omega_s$
of the shorter sections but enough for the longer section, the wave is able to
tunnel through section 1 and 2 without causing much movement and go into section
3 allowing it to experience resonance.

\subsubsection*{Perform experiment 10 and observe the results, can you determine
the beat period of tunneling.}
It would appear that the beat period of tunneling is the period of any section
of the coil in section 2 since it appears to be moving back and forth. It would
be possible to measure the period however we are running out of time.

\section{Conclusion}
The purpose of this lab was to conduct several experiments to observe multiple
properties of wave propagation through a slinky medium. The slinky was set up in
a way where we were able to observe wave behaviors when driving a piston at
various speeds.\\
Some of the notable things we determined in this lab are the following\
\begin{itemize}
    \item[-] Approximate the speed of waves through a slinky to be $c_0 = 1.536
    \pm 0.013$
    \item[-] Found that family of resonant $\omega$ values are not integer
    multiples rather they can be modeled using the equation
    $\omega = \sqrt{\omega_0^2 + k_n^2c_0^2}$ to be better
    \item[-]The amplitude of the wave appears to fall off exponentially with
    the distance away from the driven end.
    \item[-] Waves are able to tunnel through sections that are not resonant
    into areas that are resonant
\end{itemize}


\end{document}

