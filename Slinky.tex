\documentclass{article}
\usepackage{graphicx}
\usepackage[utf8]{inputenc}
\usepackage{amsmath}

\title{Investigation into wave propagation through a slinky}
\author{Rickie Li, Di Yan Xu}
\date{\today}
\begin{document}

\maketitle

\newpage

\section{Introduction}
The purpose of this experiment was to understand wave propagation through the
medium of a slinky. This was done using a slinky suspended in the air by string
and a piston moving in sinusoidal motion. The slinky being suspended in the air
by strings creates a pendulum system in which the length of the string is $l$
and the weight of the pendulum is the mass of the slinky.\\
% I got no clue what to write here maybe after its all done
The system with this slinky is a dispersive system
% somthing about the piston moving at omega

\section{Methods}
The following equipment was used for this experiment
\begin{itemize}
    \item[-] slinky
    \item[-] motor and piston
    \item[-] string
    \item[-] meter stick
    \item[-] filtering combs 
    \item[-] camera
\end{itemize}
Using setup provided we drive the motor at various different $\omega$ values
and using a camera we can find the $\omega$ by observing how long the motor
takes to make one rotation. We also use the meter stick to measure the various
features observed in the slinky as waves are propagated through.

\newpage
\section{Analysis}
\subsubsection*{Find the $c_0$ of this system.}
We know that the $c_0$ of a system is the distance travelled over time. To
approximate the $c_0$ of this system, we used the camera to capture a video
and determine the time it takes to get from one end and back. To do this we
video editing software to go frame by frame to see when the wave started and
ended. After this we determined that the wave traveled the distance of the
slinky which is a distance of $3.78 \pm 0.01$ meters in $2.46 \pm 0.02$ seconds.
Since the software used only allows for seconds to end in $0.00,0.03, 0.06$ we
estimate error to be around $0.02s$. Using this data we can approximate $c_0$
to be
\begin{align*}
    c_0 &= \frac{3.78m}{2.46s}\\
    &= 1.536 \frac{m}{s}
\end{align*}
where the error is the following
\begin{align*}
    \sigma &= c_0 \sqrt{\frac{0.01}{3.78}^2 + \frac{0.02}{2.46}^2}\\
    &= 0.013
\end{align*}
So we approximate $c_0 = 1.536 \pm 0.013 \frac{m}{s}$
\subsubsection*{Deduce $\omega_0$ with the filtering combs out of the way.}
The $\omega_0$ of a pendulum system can be determined using the following
\begin{align*}
    \omega_0 &= \sqrt{\frac{g}{L}}
\end{align*}
We measured the length of the string to be $0.875 \pm 0.005$ so we can compute
$\omega_0$
\begin{align*}
    \omega_0 &= \sqrt{\frac{9.81\frac{m}{s^2}}{0.875}}\\
    &= 3.348 s^{-1}
\end{align*}
where the error is 
\begin{align*}
    \sigma &= \omega_0 * \frac{0.005}{0.875}\\
    &= 0.019
\end{align*}
So we calculated $\omega_0 = 3.348 \pm 0.019 s^{-1}$ 


\newpage
\subsubsection*{Find the family of resonant $\omega$ values within range of
values provided by driving motor}
% my brain is lagging rn should be doing important calculations rn

\end{document}

