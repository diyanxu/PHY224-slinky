\documentclass{article}
\usepackage{graphicx}
\usepackage[utf8]{inputenc}
\usepackage{amsmath}

\title{Investigation into wave propagation through a slinky}
\author{Rickie Li, Di Yan Xu}
\date{\today}
\begin{document}

\maketitle

\newpage

\section{Introduction}
The purpose of this experiment was to understand wave propagation through the
medium of a slinky. This was done using a slinky suspended in the air by string
and a piston moving in sinusoidal motion. The slinky being suspended in the air
by strings creates a pendulum system in which the length of the string is $l$
and the weight of the pendulum is the mass of the slinky.\\
% I got no clue what to write here maybe after its all done
The system with this slinky is a dispersive system, that means the angular
frequency ($\omega$) and wave number ($k$) are releated in slightly more
complicated form than the following equations
\begin{align}
    \omega &= \sqrt{\omega_0^2 + c_0^2 k^2} \\
    k^2 &= \frac{\omega^2 - \omega_0^2}{c_0^2}
\end{align}
Where $\omega_0$ is the angular frequency for a single pendulum of the slinky
suspension string system and $c_0$ is, the constant wave speed of a
non dispersive system.\\
In this experiment we are observing different properties of wave propagation
through the slinky by running the motor at different values of $\omega$ relative
to the angular frequency of the slinky system.
% somthing about the piston moving at omega
% list some of the important -ish equations here

\section{Methods}
The following equipment was used for this experiment
\begin{itemize}
    \item[-] Slinky
    \item[-]Piston and Motor
    \item[-] String
    \item[-] Meter Stick
    \item[-] Filtering Combs 
    \item[-] Camera
\end{itemize}
Using setup provided we drive the motor at various different $\omega$ values
and using a camera we can find the $\omega$ by observing how long the motor
takes to make one rotation. We also use the meter stick to measure the various
features observed in the slinky as waves are propagated through.\\
For experiments 1 through 6, we were asked to determine and observe properties
of the slinky with no filter combs in the way. In experiments 7 through 11, We
lowered the filtering combs on certian sections of the slinky effectively
shortening the length of the string thus would increase the $\omega_0$ of the
shortened section.

\newpage
\section{Analysis}
\subsubsection*{Find the $c_0$ of this system.}
We know that the $c_0$ of a system is the distance travelled over time. To
approximate the $c_0$ of this system, we used the camera to capture a video
and determine the time it takes to get from one end and back. To do this we
video editing software to go frame by frame to see when the wave started and
ended. After this we determined that the wave traveled the distance of the
slinky which is a distance of $3.78 \pm 0.01$ meters in $2.46 \pm 0.02$ seconds.
Since the software used only allows for seconds to end in $0.00,0.03, 0.06$ we
estimate error to be around $0.02s$. Using this data we can approximate $c_0$
to be
\begin{align*}
    c_0 &= \frac{3.78m}{2.46s}\\
    &= 1.536 \frac{m}{s}
\end{align*}
where the error is the following
\begin{align*}
    \sigma &= c_0 \sqrt{\frac{0.01}{3.78}^2 + \frac{0.02}{2.46}^2}\\
    &= 0.013
\end{align*}
So we approximate $c_0 = 1.536 \pm 0.013 \frac{m}{s}$
\subsubsection*{Deduce $\omega_0$ with the filtering combs out of the way.}
The $\omega_0$ of a pendulum system can be determined using the following
\begin{align*}
    \omega_0 &= \sqrt{\frac{g}{L}}
\end{align*}
We measured the length of the string to be $0.875 \pm 0.005$ so we can compute
$\omega_0$
\begin{align*}
    \omega_0 &= \sqrt{\frac{9.81\frac{m}{s^2}}{0.875}}\\
    &= 3.348 s^{-1}
\end{align*}
where the error is 
\begin{align*}
    \sigma &= \omega_0 * \frac{0.005}{0.875}\\
    &= 0.019
\end{align*}
So we calculated $\omega_0 = 3.348 \pm 0.019 s^{-1}$ 


\newpage
\subsubsection*{Plot $\omega^2$ against $k^2$ and deduce $\omega_0^2$ and $c^2_0$
from a linear fit}
After collecting 3 data points and determining the $k$ we were able to generate
the following plot\\
% TODO: INSERT PLOT HERE
Running a curve fit on the equaiton $\omega^2, = \omega_0^2 + c_0^2k^2$, we get
a curve fit values of $\omega_0^2 = x$ and $c_0^2 = y$. These value aren't close
the the calculated values we got for $\omega_0^2$ and $c_0^2$, this may be due
to errors in data collection.


% find resonant values

\subsubsection*{Find the family of resonant $\omega$ values within range of
values provided by driving motor}
After testing different values provided by out driving motor, we determined that
the resonant values within the range of values in our driving motor are around
$x, x, x$.Looking at these values they are not integer values of some
fundanmental angular frequency, rather they are more generally releated to the
equation below for values of $n = 1,2,3$
\begin{align*}
    \omega &= \sqrt{\omega_0^2 + k_n^2c_0^2}\\
    k_n &= \frac{n\pi}{L}
\end{align*}
% my brain is lagging rn should be doing important calculations rn

\end{document}

